%!TEX root = fastZKP.tex
\section{Introduction}\label{sec:intro}


Zero-knowledge proofs (ZKP) are cryptographic protocols between two parties, a \emph{prover} and a \emph{verifier}, in which the prover can convince the verifier about the validity of a statement without leaking any extra information beyond the fact that the statement is true.  Since they were first introduced by Goldwasser et al.~\cite{goldwasser1989knowledge}, ZKP protocols have evolved from pure theoretical constructs to practical implementations, achieving proof sizes  of just hundreds of bytes and verification times of several milliseconds, regardless of the size of the statement being proved. Due to this successful transition to practice, ZKP protocols have found numerous applications not only in the traditional computation delegation setting but most importantly in providing privacy of transactions in deployed cryptocurrencies (e.g., Zcash~\cite{zerocash}) as well as in other blockchain research projects (e.g., Hawk~\cite{kosba2016hawk}). 




 




 




Despite such progress in practical implementations, ZKP protocols are still notoriously hard to scale for large statements, due to a particularly high overhead on generating the proof. For most systems, this is primarily because the prover has to  perform a large number of cryptographic operations, such as exponentiation in an elliptic curve group. And to make things worse the asymptotic complexity of computing the proof is typically more than linear, e.g., $O(C\log C)$ or even $O(C\log^2 C)$, where $C$ is the size of the statement.

Unfortunately, as of today we are yet to construct a ZKP system whose prover time is \emph{optimal}, i.e., linear in the size of the statement $C$ (this is irrespective of whether the ZKP system has per-statement trusted setup, one-time trusted setup or no trusted setup at all). The only notable exception is the recent work by B{\"u}nz et al.~\cite{bulletproofs} that however suffers from linear verification time---for a detailed comparison  see Table~\ref{table:comp}. Therefore designing ZKP systems that enjoy linear prover time as well as succinct\footnote{In ZKP literature, ``succinct" is poly-logarithmic in the size of the statement $C$.} proof size and verification time is an open problem, whose resolution can have significant practical implications. 



\paragraph{Our contributions.} In this paper we propose \name, the first ZKP protocol with \emph{linear prover time} and \emph{succinct proof size and verification time} in the size of the arithmetic circuit representing the statement $C$\footnote{The verification time is succinct in our protocol only when the circuit is log-space uniform. For all ZKP protocols without per-circuit setup, this is the best that can be achieved, as the verifier must read the entire circuit, which takes linear time in the worst case. We always refer to log-space uniform circuits when we say our scheme is succinct in this paper, to differentiate from schemes with linear verification time on all circuits (irrespective of whether the circuits are log-space uniform or not). Schemes such as \textsf{libSTARK}~\cite{libstark}, \textsf{zkVSQL}~\cite{zkvpd} and \textsf{Hyrax}~\cite{hyrax} also have such property. In practice, most computations can be expressed as log-space uniform circuits. Moreover, as shown in~\cite{libsnark,vram,libstark}, all RAM programs  can be compiled into circuits that are log-space uniform in the running time of the programs (but linear in the size of the programs).}. \name is based on the doubly efficient interactive proof protocol proposed by Goldwasser et al. in~\cite{GKR} (referred as GKR protocol in this paper), and the verifiable polynomial delegation scheme proposed by Zhang et al. in~\cite{zhang2017vsql}. As such it comes with \emph{one-time trusted} setup (and not per-statement trusted setup) that depends only on the size of the input (witness) to the statement that is being proved. Not only does \name have excellent asymptotics but also its prover outperforms in practice all other ZKP systems while verification time and proof size are also very competitive---see Table~\ref{table:comp}. Our concrete contributions are:
\begin{itemize}[leftmargin=*]
	\item \textbf{GKR with linear prover time.} \name features a new linear-time algorithm to generate a GKR proof. Our new algorithm does not require any pattern in the circuit and our result subsumes all existing improvements on the GKR prover assuming special circuit structures, such as regular circuits in~\cite{JT_Thesis}, data-parallel circuits in~\cite{JT_Thesis,wahby2017full}, circuits with different sub-copies in~\cite{vram}. See related work for more details. 
	\item \textbf{Adding zero-knowledge.} We propose an approach to turn \name into zero-knowledge efficiently. In particular, we show a way to mask the responses of our linear-time prover with small random polynomials such that the zero-knowledge variant of the protocol introduces minimal overhead on the verification time compared to the original (unmasked) construction. 
	\item \textbf{Implementation and evaluation.}  We implement \name. Our implementation takes an arithmetic circuit with various types of gates (fan-in 2 and degree $\le 2$, such as $+,-,\times$, AND, XOR, etc.) and compiles it into a ZKP protocol. We conduct thorough comparisons to all existing ZKP systems (see Section~\ref{comp-sec}). We plan to release our system as an open-source implementation.

\end{itemize}

\subsection{Comparing to other ZKP Systems\label{comp-sec}} Table~\ref{tab:zkpall} shows a detailed comparison between \name and existing ZKP systems. First of all, \name is the best among all existing systems in terms of practical prover time. In terms of asymptotics, \name is the only system with linear prover time and succinct verification and proof size for log-space uniform circuits.
%Log-space uniform circuits can express a large class of computations. In particular, as it is shown in~\cite{libsnark,vram,libstark}  random access machine computations can be reduced to log-space uniform circuits. 
The only other system with linear prover time is \textsf{Bulletproofs}~\cite{bulletproofs} whose verification time is linear, \emph{even for log-space uniform circuits}. In the practical front, \textsf{Bulletproofs} prover time and verification time are high, due to the large number of cryptographic operations required for every gate of the circuit. 

The proof  and verification  of \name are also competitive to other systems. In asymptotic terms, our proof size is only larger than \textsf{libSNARK}~\cite{libsnark} and \textsf{Bulletproofs}~\cite{bulletproofs}, and our verification is slower than \textsf{libSNARK}~\cite{libsnark} and \textsf{libSTARK}~\cite{libstark}. Compared to \textsf{Hyrax}~\cite{hyrax}, which is also based on similar techniques with our work, \name improves the performance in all aspects (yet \textsf{Hyrax} does not have any trusted setup). One can refer to Section~\ref{sec:eval} for a detailed  description of our experimental setting as well as a more detailed comparison. 
 

 
Finally, among all systems, \textsf{libSNARK}~\cite{libsnark} requires a trusted setup for every statement, and \name requires an one-time trusted setup that depends on the input size. See Section~\ref{subsec::discuss} for a discussion on removing trusted setup in \name. 

 %\dawn{we may consider adding a row in the table to show the trusted setup, e.g., making it explicit in the table about the difference in trusted setup btw SNARK and Buggati.} 



\begin{table}[t]
\caption{\label{table:comp}Comparison of \name to existing ZKP systems, where $(\mathcal{G},\mathcal{P},\mathcal{V},|\pi|)$ denote the trusted setup algorithm, the prover algorithm, the verification algorithm and the proof size respectively. Also, $C$ is the size of the log-space uniform circuit with depth $d$, and $n$ is the size of its input. The numbers are for a circuit computing the root of a Merkle tree with 256 leaves (511 instances of SHA256).\protect\footnotemark}\label{tab:zkpall}

	\centering
	{\fontsize{8}{8}
	\begin{tabular}{|c|c|c|c|c|c|c|c|}
		
		\hline
		&\textsf{libSNARK}&\textsf{Ligero}&\textsf{Bulletproofs}&\textsf{Hyrax}&\textsf{libSTARK}&\textsf{Aurora}&\name\\
		&\cite{libsnark}&\cite{ligero}&\cite{bulletproofs}&\cite{hyrax}&\cite{libstark}&\cite{aurora}&\\
		\hline
		\hline
		&$O(C)$&\multicolumn{5}{c|}{}&$O(n)$\\
		$\mathcal{G}$&\textsf{per-statement}&\multicolumn{5}{c|}{\textsf{no trusted setup}}&\textsf{one-time}\\
		&\textsf{trusted setup}&\multicolumn{5}{c|}{}&\textsf{trusted setup}\\
		\hline
		$\P$&$O(C\log^2C)$&$O(C\log C)$&$O(C)$&$O(C\log C)$&$O(C\log^2 C)$&$O(C\log C)$ &$O(C)$\\
		\hline
		$\V$&$O(1)$&$O(C)$&$O(C)$&$O(\sqrt{n}+d\log C)$&$O(\log^2 C)$&$O(C)$&$O(d\log C)$\\
		\hline
		$|\pi|$&$O(1)$&$O(\sqrt{C})$&$O(\log C)$&$O(\sqrt{n}+d\log C)$&$O(\log^2 C)$& $O(\log^2 C)$&$O(d\log C)$\\
		\hline
		\hline
		$\mathcal{G}$&1027s&\multicolumn{5}{c|}{NA}&210s\\
		\hline
		$\P$&360s&400s&13,000s&1,041s&30,000s&3199s&201s\\
		\hline
			$\V$&0.002s&4s&900s&9.9s&0.02s&15.2s&0.71s\\
		\hline
		$|\pi|$&0.13KB&1,500KB&5.5KB&185KB&728KB&174.3KB&51KB\\
		\hline
	\end{tabular}
}

\end{table}
\footnotetext{\textsf{STARK} is in the RAM model. To compare the performance, we convert a circuit of size $C$ to a RAM program with $T=\Theta(C)$ steps. }

\subsection{Our Techniques}
Our main technical contributions are a GKR protocol with linear prover time and an efficient approach to turn the GKR protocol into zero-knowledge. We summarize the key ideas behind these two contributions. The detailed protocols are presented in Section~\ref{sec::gkrlin} and~\ref{sec:zkp} respectively.

\paragraph{GKR with linear prover.} Goldwasser et al.~\cite{GKR} showed an approach to model the evaluation of a layered circuit as a sequence of summations on polynomials defined by values in consecutive layers of the circuit. Using the famous sumcheck protocol (see Section~\ref{subsec::sumcheck}), they developed a protocol (the GKR protocol) allowing the verifier to validate the circuit evaluation in logarithmic time with a logarithmic size proof. However, the polynomials in the protocol are multivariate with $2s$ variables, where $S$ is the number of gates in one layer of the circuit and $s = \log S$. Naively running the sumcheck protocol on these polynomials incurs $S^2$ prover time, as there are at least $2^{2s}=S^2$ monomials in a $2s$-variate polynomial. Later, Cormode et al.~\cite{CMT} observed that these polynomials are sparse, containing only $S$ nonzero monomials and improved the prover time to $S\log S$.

In our new approach, we divide the protocol into two separate sumchecks. In each sumcheck, the polynomial only contains $s$ variables, and can be expressed as the product of two multilinear polynomials. Utilizing the sparsity of the circuit, we develop new algorithms to scan through each gate of the circuit and compute the closed-form of all these multilinear polynomials explicitly, which takes $O(S)$ time. With this new way of representation, the prover can deploy a dynamic programming technique to generate the proofs in each sumcheck in $O(S)$ time, resulting in a total prover time of $O(S)$. 

\paragraph{Efficient zero-knowledge GKR.} The original GKR protocol is not zero-knowledge, since the messages in the proof can be viewed as weighed sums of the values in the circuit and leak information. In~\cite{zkvpd,hyrax}, the authors proposed to turn the GKR protocol into zero-knowledge by hiding the messages in homomorphic commitments, which incurs a big overhead in the verification time. In~\cite{zksumcheck}, Chiesa et al. proposed an alternative approach by masking the protocol with random polynomials. However, the masking polynomials are as big as the original ones and the prover time becomes exponential, making the approach mainly of theoretical interest. 

In our scheme, we first show that in order to make the sumcheck protocol zero-knowledge, the prover can mask it with a ``small" polynomial. In particular, the masking polynomial only contains logarithmically many random coefficients. The intuition is that though the original polynomial has $O(2^\ell)$ or more terms ($\ell$ is the number of variables in the polynomial), the prover only sends $O(\ell)$ messages in the sumcheck protocol. Therefore, it suffices to mask the original polynomial with a random one with $O(\ell)$ coefficients to achieve zero-knowledge\babis{what if I do two separate executions of GKR with two different users and then the users compose their proofs? Would not that reveal more messages and therefore zero-knowledge might stop working?}\yupeng{There will be two different masks.}. In particular, we set the masking polynomial as the sum of $\ell$ univariate random polynomials with the same variable-degree. In Section~\ref{subsec:zksumcheck}, we show that the entropy of this mask exactly counters the leakage of the sumcheck, proving that it is sufficient and optimal.

Besides the sumcheck, the GKR protocol additionally leaks two evaluations of the polynomial defined by values in each layer of the circuit. To make these evaluations zero-knowledge, we mask the polynomial by a special low-degree random polynomial. In particular, we show that after the mask, the verifier in total learns 4 messages related to the evaluations of the masking polynomial and we can prove zero-knowledge by making these messages linearly independent. Therefore, the masking polynomial is of constant size: it consists of 2 variables with variable degree 2.


\subsection{Related Work}\label{subsec::related}
\babis{LOW PRIORITY, BUT WOULD BE GOOD IF YOU CAN ADDRESS: The related work has a lot of repetitions with the part of the paper before ``our new techniqoues". this can be annoying for the reviewer. Whatever we say for systems in related work should be said once. I believe we mention three times in the intro that Bulletproofs has linear verification time. SHOULD BE COMPRESSED---checkout my CRYPTO 2018 paper, I was very careful not to repeat anything in the intro---sometimes having a seperate related work section is annoying because it makes you repeat what you said to motivate the paper}
 In recent years there has been significant progress in efficient ZKP protocols and systems. In this section, we discuss related work in this area, with the focus on those with sublinear proofs. 

\paragraph{QAP-based.} Following earlier work of Ishai~\cite{IshaiKO07}, Groth~\cite{groth2010short} and Lipmaa~\cite{lipmaa2012progression}, Gennaro et al.~\cite{GGPR13} introduced quadratic arithmetic programs (QAPs), which forms the basis of most recent implementations~\cite{parno2013pinocchio,ben2013snarks,braun13,ben2014scalable,geppetto,DBLP:conf/eurocrypt/ChiesaTV15,wahby2015efficient,Fiore16,wu2018dizk} including \textsf{libSNARK}~\cite{libsnark}. The proof size in these systems is constant, and the verification time depends only on the input size. Both these properties are particularly appealing and have led to real-world deployments, e.g., ZCash~\cite{zerocash}. One of the main bottlenecks, however, of QAP-based systems  is the high overhead in the prover running time and memory consumption, making it hard to scale to large statements. In addition, a separate trusted setup for every different statement is required. 

\paragraph{STARKs.} Following the seminal work of Kilian~\cite{Kilian92} and Micali~\cite{Micali00}, Ben-Sasson et al. built \textsf{libstark}~\cite{libstark} a zero-knowledge transparent argument of knowledge (zkSTARK) using short probabilistically checkable proofs (PCPs). \textsf{libSTARK} does not rely on trusted setup and execute in the RAM model of computation. Their verification time is only linear to the description of the RAM program, and succinct (logarithmic) in the time required for program execution.  However, due to the heavy machinery of PCPs, the prover time and memory usage is large in practice--- see Section~\ref{sec:eval}. 

\paragraph{IOPs.} Based on ``(MPC)-in-the-head" introduced in~\cite{ishai2007zero,giacomelli2016zkboo,chase2017post}, Ames et al.~\cite{ligero} proposed a ZKP scheme called \textsf{Ligero}. It only uses symmetric key operations and the prover time is fast in practice. However, it generates proofs of size $O(\sqrt{C})$, which is several megabytes in practice for moderate-size circuits. In addition, the verification time is quasi-linear to the size of the circuit. It is categorized as interactive PCP, which is a special case of interactive oracle proofs (IOPs). Recently, Ben-Sasson et al.~\cite{aurora} proposed \textsf{Aurora}, a new ZKP system in the IOP model which improves the proof size to $O(\log^2 C)$.   

\paragraph{Discrete log.} Before Bulletproof~\cite{bulletproofs}, earlier discrete-log based ZKP schemes include the work of Groth~\cite{groth2009linear}, Bayer and Groth~\cite{bayer2012efficient} and Bootle et al.~\cite{bootle2016efficient,bootle2017linear}. The proof size of these schemes are larger than Bulletproof either asymptotically or concretely.


% Building on the work of Groth~\cite{groth2009linear}, Bayer and Groth~\cite{bayer2012efficient} and Bootle et al.~\cite{bootle2016efficient,bootle2017linear}, B{\"u}nz et al.~\cite{bulletproofs} proposed \textsf{Bulletproofs}, a ZKP system based on discrete logarithm assumptions. To the best of our knowledge, this is the only line of work that achieves a linear prover time. The proof size in \textsf{Bulletproofs} is $O(\log C)$, and is small in practice (several KBs). However, both the prover and the verification involve several cryptographic operations per gate of the circuit, limiting the circuit size that  can be handled.


\paragraph{Interactive proofs.}\yupeng{This can be further shortened. However, we said in the contributions that we will explain more in related work.} The line of work that relates to our paper the most is based on interactive proofs~\cite{goldwasser1989knowledge}. In the seminal work of~\cite{GKR}, Goldwasser et al. proposed an efficient interactive proof for layered arithmetic circuits. Later, Cormode et al.~\cite{CMT} improved the prover complexity of the interactive proof in~\cite{GKR} to $O(C\log C)$ using multilinear extensions instead of low degree extensions. Several follow-up works further reduce the prover time assuming special structures of the circuit. For regular circuits where the wiring pattern can be described in constant space and time, Thaler~\cite{JT_Thesis} introduced a protocol with $O(C)$ prover time; for data parallel circuits with many copies of small circuits with size $C'$, a $O(C\log C')$ protocol is presented in the same work, later improved to $O(C+C'\log C)$ by Wahby et al. in~\cite{wahby2017full}; for circuits with many non-connected but different copies, Zhang et al. showed a protocol with $O(C\log C')$ prover time.

In~\cite{zhang2017vsql}, Zhang et al. extended the GKR protocol to an argument system using a protocol for verifiable polynomial delegation. Zhang et al.~\cite{vram} and Wahby et al.~\cite{hyrax} make the argument system zero-knowledge by putting all the messages in the proof into homomorphic commitments, as proposed by Cramer and Damgard in~\cite{cramer1998zero}. This approach introduces a high overhead on the verification time compared to the plain argument system without zero-knowledge, as each addition becomes a multiplication and each multiplication becomes an exponentiation in the homomorphic commitments. The multiplicative overhead is around two orders of magnitude in practice. Additionally, the scheme of~\cite{hyrax}, \textsf{Hyrax}, removes the trusted setup of the argument system by introducing a new polynomial delegation, increasing the proof size and verification time to $O(\sqrt{n})$ where $n$ is the input size of the circuit. 

\paragraph{Lattice-based.} Recently Baum et al.~\cite{baum2018sub} proposed the first lattice-based ZKP system with sub-linear proof size. The proof size is $O(\sqrt{C\log^3C})$, and the practical performance is to be explored.