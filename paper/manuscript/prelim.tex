%!TEX root = fastZKP.tex
\section{Preliminary}
\label{sec::prelim}

\subsection{Notations}

polynomials, PPT, vectors, ,binary, array, security parameter, $\neg$,

\subsection{Cryptographic Assumptions}

\paragraph{Bilinear pairing.} Let $\lambda$ be the security parameter, $\mathbb{G}, \mathbb{G}_T$ be two groups of order $p$ and $g\in\mathbb{G}$ be a generator. $e: \mathbb{G}\times\mathbb{G}\rightarrow\mathbb{G}_T$ denotes a bilinear map and we use $\mathsf{bp}=(p,\mathbb{G},\mathbb{G}_T,e,g)\leftarrow\mathsf{BilGen}(1^\lambda)$ for the generation of parameters for the bilinear map.

Our scheme relies on the following assumptions.

\begin{assumption}[$q$-Strong Diffie-Hellman]
	\label{asp::qSDH}
	For any probabilistic polynomial time (PPT) adversary $\mathcal{A}$, the following holds:
	
	\[\Pr\left[ \begin{aligned}
	&\mathsf{bp} \leftarrow \mathsf{BilGen}(1^\lambda) & \\
	&s \overset{R}{\leftarrow} \mathbb{Z}^{*}_{p} & : (x, e(g, g)^{\frac{1}{s+x}}) \leftarrow \mathcal{A}(1^\lambda, \sigma)\\
	&\sigma = (\mathsf{bp}, g^s, ..., g^{s^q})
	\end{aligned} \right] \le \negl{(\lambda)}\]
\end{assumption}

The second assumption is a generalization of the $q$-PKE assumption~\cite{groth2010short} to multivariate polynomials, proposed in~\cite{zhang2017vsql,zkvpd}. Let $\mathcal {W}_ {\ell,d}$ be the set of all multisets of $\{1, . . . , \ell\}$ with the cardinality of each element being at most $d$. 

\begin{assumption}[$(d,\ell)$-Extended Power Knowledge of Exponent]
	\label{asp::dlEPKE}
	For any PPT adversary $\mathcal{A}$, there is a polynomial time algorithm $\mathcal{E}$ (takes the same randomness of $\mathcal{A}$ as input) such that for all benign auxiliary inputs $z \in \{0,1\}^{\poly({\lambda})}$ the following probability is negligible:
	\[\Pr\left[ \begin{aligned}
	\mathsf{bp} \leftarrow \mathsf{BilGen}(1^\lambda) && \\
	s_1, ..., s_\ell, s_{\ell + 1}, \alpha \overset{R}{\leftarrow} Z_{p}^*, s_0=1 && \\
	\sigma_1 = (\{g^{\prod_{i\in W}s_i}\}_{W\in \mathcal{W}_{\ell, d}, g^{s_{\ell+1}}}) &&\hspace*{10pt} e(h, g^\alpha)=e(\tilde{h}, g)\\
	\sigma_2 = (\{g^{\alpha\prod_{i\in Ws_i}}\}_{W\in \mathcal{W}_{\ell, d}}, g^{\alpha s_{\ell+1}}) && : \prod_{W\in \mathcal{W}_{\ell, d}}g^{a_{W}\prod_{i\in W}s_i}g^{b{s_{\ell+1}}}\neq h\\
	\sigma = (\mathsf{bp}, \sigma_1, \sigma_2, g^{\alpha}) && \\
	\mathbb{G} \times \mathbb{G} \ni (h, \tilde{h}) \leftarrow \mathcal{A} \left(1^{\lambda}, \sigma, z\right)&& \\
	\left( a _ { 0 } , \dots , a _ { \left| \mathcal { W } _ { \ell , d } \right| } , b \right) \leftarrow \mathcal { E } \left( 1 ^ { \lambda } , \sigma , z \right) &&
	\end{aligned}\right] \le \negl{(\lambda)}\]
\end{assumption}

\subsection{Interactive Proof and Zero-knowledge Arguments}

\paragraph{Interactive proof.} Traditional proof involves two static objects: a prover $\mathcal{P}$ and a verifier $\mathcal{V}$. The prover $\mathcal{P}$ takes a statement $x$ as input and generate a string $\pi$ as proof, then the verifier $\mathcal{V}$ checks if the statement $x$ and the proof $\pi$ are correct. An interactive proof is a stronger notion of proof, it allows the prover $\mathcal{P}$ to convince the verifier $\mathcal{V}$ of the validity of some statement. The interactive proof runs in several rounds, allowing the verifier to ask questions in each round based on prover's answers of previous rounds. We phrase this in terms of $\mathcal{P}$ trying to convince $\mathcal{V}$ that $f(x)=1$. The proof system is interesting if and only if the running time of $\mathcal{V}$ is less than the time of directly computing the function $f$.

We formalize the "interactive proof" in the following:	
\begin{definition}\label{def:ip}
	Let $f$ be a boolean function. A pair of interactive machines $\langle\mathcal{P}, \mathcal{V}\rangle$ is an interactive proof for $f$ with soundness $\epsilon$ if the following holds:
	\begin{itemize}
		\item {\bf Completeness.} For every $x$ such that $f(x) = 1$ it holds that $\Pr[\langle\mathcal{P}, \mathcal{V}\rangle(x)=accept]=1$.
		\item {\bf $\epsilon$-Soundness.} For any $x$ with $f(x) \neq 1$ and any $\mathcal{P}^*$ it holds that $\Pr[\langle\mathcal{P^*},\mathcal{V}\rangle=accept] \le \epsilon$
	\end{itemize}
\end{definition}


\paragraph{Zero-knowledge Argument.} An argument system for an NP relationship $R$ is a protocol between a computationally bounded prover $\P$ and a verifier $\V$. At the end of the protocol, $\V$ is convinced by $\P$ that there exists a witness $w$ such that $(x; w) \in R$ for some input $x$. We focus on arguments of knowledge which have the stronger property that if the prover convinces the verifier of the statement’s validity, then the prover must know $w$. We use $\mathcal{G}$ to represent private key $\pk$ and verificaiton key $\vk$ generation phase. Formally, consider the definition below.

\begin{definition}\label{def::zkp}
	
	Let $R$ be an NP relation and let $\lambda$ be the security parameter. A tuple of algorithm $(\mathcal{G}, \mathcal{P}, \mathcal{V})$ is a zero knowledge argument fro $R$ if the following holds.
	
	\begin{itemize}
		
		\item \textbf{Correctness}. For every $(\pk, \vk)$ output by $\mathcal{G}(1^\lambda)$ and $(x, w) \in R$, 
		$$\langle \P(\pk, w), \V(\vk) \rangle(x) = \accept$$
		
		\item \textbf{Soundness}. For any PPT prover $\P$, there exists a PPT extractor $\varepsilon$ such that for any $x$ it holds that
		
		$$\Pr[\langle\P(\pk), \V(\vk) \rangle(x) = \accept \wedge (x, w) \notin R | w \leftarrow \varepsilon(\pk, x)] \leq \neg(\lambda)$$
		
		\item \textbf{Zero knowledge}. There exists a PPT simulator $\S$ such that for any PPT adversary $\A$, auxiliary input $z \in \{0, 1\}^{poly(\lambda)}$, it holds that
		
		$$\Pr\left[(x;w)\in R; \langle\P(\pk,w),\A(\sigma) \rangle=\accept: (\pk,\vk)\leftarrow\mathcal{G}(1^\lambda); (x,w,\sigma)\leftarrow\A(z,\pk,\vk) \right] = $$
		$$\Pr\left[(x;w)\in R; \langle\S(\mathsf{trap}, z, \pk),\A(\sigma) \rangle=\accept:(\pk,\vk,\mathsf{trap})\leftarrow\S(1^\lambda); (x,w,\sigma)\leftarrow\A(z,\pk,\vk)\right]$$
		
		where $=$ means perfect zero knowledge. 
		
	\end{itemize}
	
\end{definition}

%!TEX root = fastZKP.tex

\subsection{GKR Protocol}\label{subsec::GKR}

In~\cite{GKR}, Goldwasser et al. proposed an efficient interactive proof protocol for layered arithmetic circuits, which we use as a building block for our new zero-knowledge argument and is referred as the \emph{GKR} protocol. It is later improved in~\cite{CMT,t13, wahby2017full,vram} for different categories of circuits. We present the detailed protocol here.



\subsubsection{Sumcheck Protocol.}
\label{subsec::sumcheck}
The sumcheck problem is a fundamental problem that serves as a building block for varies applications. The problem is to sum a polynomial $f: \mathbb{F}^\ell \rightarrow \mathbb{F}$ on the binary hypercube $$\sum\limits_{b_1,b_2,\ldots,b_\ell\in\{0,1\}}f(b_1,b_2,...,b_\ell).$$ 
Directly computing the sum requires exponential time in $\ell$, as there are $2^\ell$ combinations of $b_1,\ldots,b_\ell$. Lund et al.~\cite{sumcheck} proposed a \emph{sumcheck} protocol that allows a verifier $\mathcal{V}$ to delegate the computation to a computationally unbounded prover $\mathcal{P}$, who can convince $\mathcal{V}$ that $H$ is the correct sum. We provide a description of the sumcheck protocol in Protocol~\ref{prot::sumcheck}.
\begin{figure}[t!]
\small{
\centering{\centering
\framebox{\parbox{.99\linewidth}{
\begin{protocol}[\textbf{Sumcheck}]
	\label{prot::sumcheck}
	The protocol proceeds in $\ell$ rounds. 
	\begin{itemize}
		\item In the first round, $\P$ sends a univariate polynomial $$f_1(x_1)\overset{def}{=}\sum\limits_{b_2,\ldots,b_\ell\in\{0,1\}}f(x_1,b_2,...,b_\ell)\, ,$$ $\V$ checks $H=f_1(0)+f_1(1)$. Then $\V$ sends a random challenge $r_1\in\mathbb{F}$ to $\P$.
		\item In the $i$-th round, where $2\le i \le l-1$, $\P$ sends a univariate polynomial
		$$f_{i}(x_{i})\overset{def}{=}\sum\limits_{b_{i+1},\ldots,b_\ell\in\{0,1\}}f(r_1,\ldots, r_{i-1}, x_{i}, b_{i+1},\ldots, b_{\ell})\, ,$$ 
		$\V$ checks $f_{i-1}(r_{i-1})=f_{i}(0)+f_{i}(1)$, and sends a random challenge $r_{i}\in\mathbb{F}$ to $\P$.
		\item In the $\ell$-th round, $\P$ sends a univariate polynomial $$f_{\ell}(x_{\ell})\overset{def}{=}f(r_1, r_2, ..., r_{l-1}, x_{\ell})\, ,$$ $\V$ checks $f_{\ell-1}(r_{\ell-1})=f_{\ell}(0)+f_{\ell}(1)$. The verifier generates a random challenge $r_{\ell}\in\mathbb{F}$. Given oracle access to an evaluation $f(r_1, r_2, ..., r_\ell)$ of $f$, $\V$ will accept if and only if $f_{\ell}(r_\ell) = f(r_1, r_2, ..., r_\ell)$. The instantiation of the oracle access depends on the application of the sumcheck protocol.
	\end{itemize}
\end{protocol}}}}}
\end{figure}
The proof size of the sumcheck protocol is $O(d\ell)$, where $d$ is the variable-degree of $f$, as in each round, $\P$ sends a univariate polynomial for each variable in $f$, which can be uniquely defined by $d+1$ points. The verifier time of the protocol is $O(d\ell)$. The prover time depends on the degree and the sparsity of $f$, and we will give the complexity later in our scheme. The sumcheck protocol is complete and sound with $\epsilon = \frac{d\ell}{|\mathbb{F}|}$. 


\begin{definition}[\textbf{Multi-linear Extension}]
	Let $V:\{0, 1\}^\ell \rightarrow \mathbb{F}$ be a function. The \textit{multilinear extension} of $V$ is the unique polynomial $\tilde{V}: \mathbb{F}^l \rightarrow \mathbb{F}$ such that $\tilde{V}(x_1, x_2, ..., x_{l}) = V(x_1, x_2, ..., x_{l})$ for all $x_1, x_2, ..., x_{l}\in\{0,1\}^l$.
	
	
	$\tilde{V}$ can be expressed as:
	$$\tilde{V}(x_1, x_2, ..., x_{l})=\sum_{b\in\{0,1\}^l}\prod_{i=1}^{l}[((1-x_i)(1-b_i)+x_ib_i) \times V(b)]$$
	where $b_i$ is $i$-th bit of b.
	
	
\end{definition}

\paragraph{Multilinear extensions of arrays.} Inspired by the close form equation of the multilinear extension given above, we can view an array $\textbf{A} = (a_0, a_1, \ldots, a_{n-1})$ as a function $A: \binary^{\log n}\rightarrow \F$ such that $\forall i\in[0,n-1], A(i) = a_i$. Therefore, in this paper, we abuse the use of multilinear extension on an array as the multilinear extension $\tilde{A}$ of $A$.    

Using the sumcheck protocol as a building block, Goldwasser et al.~\cite{GKR} showed an interactive proof protocol for layered arithmetic circuits. 

\smallskip\noindent\textbf{High Level Ideas.} Let $C$ be a layered arithmetic circuit with depth $d$ over a finite field $\mathbb{F}$. Each gate in the $i$-th layer takes inputs from two gates in the $(i+1)$-th layer; layer $0$ is the output layer and layer $d$ is the input layer. The protocol proceeds layer by layer. Upon receiving the claimed output from $\P$, in the first round, $\V$ and $\P$ run the sumcheck protocol to reduce the claim about the output to a claim about the values of the wire in the layer above. In the $i$-th round, both parties reduces a claim about layer $i-1$ to a claim about layer $i$ through the sumcheck protocol. Finally, the protocol terminates with a claim about the input layer $d$, which can be checked directly by $\V$, or given as an oracle access. If the check passes, $\V$ accepts the claimed output. 

\paragraph{Notations.} Before describing the GKR protocol, we introduce some additional notations. We denote the number of gates in the $i$-th layer as $S_i$ and let $s_i = \ceil{\log S_i}$. (For simplicity, we assume $S_i$ is a power of 2, and we can pad the layer with dummy gates otherwise.) We then define a function $V_i:\binary^{s_i}\rightarrow\mathbb{F}$ that takes a binary string $b\in\binary^{s_i}$ and returns the output of gate $b$ in layer $i$, where $b$ is called the gate label. With this definition, $V_0$ corresponds to the output of the circuit, and $V_d$ corresponds to the input layer. Finally, we define two additional functions $add_i, mult_i: \binary^{s_{i-1}+2s_i}\rightarrow\binary$, referred as \emph{wiring predicates} in the literature. $add_i$ ($mult_i$) takes one gate label $z\in\binary^{s_{i-1}}$ in layer $i-1$ and two gate labels $x,y\in\binary^{s_i}$ in layer $i$, and outputs 1 if and only if gate $z$ is an addition (multiplication) gate that takes the output of gate $x,y$ as input. With these definitions, $V_i$ can be written as follows:
\begin{equation}
    \begin{aligned}
	V_i(z)=\sum_{x, y \in\binary^{s_{i+1}}}(add_{i+1}(z,x,y)(V_{i+1}(x)+V_{i+1}(y))\\
	+mult_{i+1}(z,x,y)(V_{i+1}(x)V_{i+1}(y)))
	\end{aligned}
\end{equation}
for any $z\in\binary^{s_i}$. 









\ignore{
\yupeng{try to remove $\beta$.}

\begin{definition}[Multilinear extension of identity function]
Here we present a multilinear extension of identity function
\begin{eqnarray}\beta_{l}(a, b)=
	\begin{cases}
	1, &a=b\cr 0, &a\neq b
	\end{cases}
\end{eqnarray}
Where $a, b$ are binary strings with length $l$. The multilinear extension is the following:

$$\tilde{\beta_{l}}(a,b)\overset{def}{=}\prod_{j=1}^{l}((1-a_{j})(1-b_{j})+a_{j}b_{j})$$

We have $\tilde{\beta_{l}}(a,b)=\beta_{l}(a, b)$ when $a, b$ are binary.
\end{definition}

\begin{definition}[Multilinear extension of $add$, $mult$]
We will define the multilinear extension of two component of $V_i$.
$$\tilde{add}_{i}(g, u, v)=\sum_{(g', u', v') \in G_{i, add}}\tilde{\beta}_{s_i}(g, g')\tilde{\beta}_{s_{i+1}}(u, u')\tilde{\beta}_{s_{i+1}}(v, v')$$

$$\tilde{mult}_{i}(g, u, v)=\sum_{(g', u', v') \in G_{i, mult}}\tilde{\beta}_{s_i}(g, g')\tilde{\beta}_{s_{i+1}}(u, u')\tilde{\beta}_{s_{i+1}}(v, v')$$
\end{definition}

\begin{definition}[Multilinear extension of $V_i(g)$]
	\label{def::multilinear}
	%$$\tilde{V}_{i}(z)=\sum_{g\in\{0,1\}^{s_i} u, v\in \{0,1\}^{s_{i+1}}}f_{i,z}(g,u,v)$$

	%$$f_{i,z}(g,u,v)\overset{def}{=}\tilde{\beta_{i}}(z, g)[\tilde{mult}(g, u, v)(\tilde{V}_{i+1}(u)\tilde{V}_{i+1}(v))+\tilde{add}(g,u,v)(\tilde{V}_{i+1}(u)+\tilde{V}_{i+1}(v))]$$
	%where $$\tilde{\beta_{i}}(z,g)\overset{def}{=}\prod_{j=1}^{s_{i}}((1-g_{j})(1-z_{j})+g_{j}z_{j})$$

	%$$\tilde{mult}(g,u,v)\overset{def}{=}\sum_{(g', u', v')\in G_{i,mult}}\tilde{\beta_{i}}(g,g')\tilde{\beta_{i+1}}(u,u')\tilde{\beta_{i+1}}(v,v')$$

	%$$\tilde{add}(g,u,v)\overset{def}{=}\sum_{(g', u', v')\in G_{i,add}}\tilde{\beta_{i}}(g,g')\tilde{\beta_{i+1}}(u,u')\tilde{\beta_{i+1}}(v,v')$$

	%when $a, b$ are binary strings, $\tilde{\beta_{i}}(a, b)$ will output $1$ when $a=b$, and $0$ otherwise. $\tilde{mult},\tilde{add}$ is the multilinear extension of the wiring predicates. A wiring predicates is a function of three gates $g_1, g_2, g_3$, and returns $1$ if $g_1$ takes $g_2, g_3$ as it's input.

	We define the multilinear extension of $V_i(g)$ in the following way:

	$$\tilde{V}_i(g)=\sum_{u, v \in \{0,1\}^{s_{i+1}}}(\tilde{add}_{i+1}(g, u, v)(\tilde{V}_{i+1}(u)+\tilde{V}_{i+1}(v))+\tilde{mult}_{i+1}(g,u,v)(\tilde{V}_{i+1}(u)\tilde{V}_{i+1}(v)))\textnormal{,}$$
	where $\tilde{mult}$ and $\tilde{add}$ are multilinear extension of $mult$ and $add$.
\end{definition}

}


In the equation above, $V_i$ is expressed as a summation, so $\V$ can use the sumcheck protocol to check that it is computed correctly. As the sumcheck protocol operates on polynomials defined on $\mathbb{F}$, we rewrite the equation with their multilinear extensions:

\begin{align}\label{eq:GKR}
\tV_i(g)=&\sum_{x, y \in\binary^{s_{i+1}}}f_i(x,y)\nonumber\\
=&\sum_{x, y \in\binary^{s_{i+1}}}(\tadd_{i+1}(g,x,y)(\tV_{i+1}(x)+\tV_{i+1}(y))\nonumber\\
&+\tmult_{i+1}(g,x,y)(\tV_{i+1}(x)\tV_{i+1}(y)))
\end{align}
where $g\in\mathbb{F}^{s_i}$ is a random vector. 

\paragraph{Protocol.} With Equation~\ref{eq:GKR}, the GKR protocol proceeds as following. The prover $\P$ first sends the claimed output of the circuit to $\V$. From the claimed output, $\V$ defines polynomial $\tV_0$ and computes $\tV_0(g)$ for a random $g\in\mathbb{F}^{s_0}$. $\V$ and $\P$ then invoke a sumcheck protocol on Equation~\ref{eq:GKR} with $i=0$. As described in Section~\ref{subsec::sumcheck}, at the end of the sumcheck, $\V$ needs an oracle access to $f_i(u,v)$, where $u,v$ are randomly selected in $\mathbb{F}^{s_{i+1}}$. To compute $f_i(u,v)$, $\V$ computes $\tadd_{i+1}(u,v)$ and $\tmult_{i+1}(u,v)$ locally (they only depend on the wiring pattern of the circuit, but not on the values), asks $\P$ to send $\tV_1(u)$ and $\tV_1(v)$ and computes $f_i(u,v)$ to complete the sumcheck protocol. In this way, $\V$ and $\P$ reduces a claim about the output to two claims about values in layer 1. $\V$ and $\P$ could invoke two sumcheck protocols on $\tV_1(u)$ and $\tV_1(v)$ recursively to layers above, but the number of claims and the sumcheck protocols increases exponentially in $d$. 

\smallskip\noindent\textbf{Combining two claims: condensing to one claim.} In~\cite{GKR}, Goldwasser et al. presented a protocol to reduce two claims $\tV_i(u)$ and $\tV_i(v)$ to one as following. $\V$ defines a line $\gamma: \mathbb{F} \rightarrow \mathbb{F}^{s_i}$ such that $\gamma(0)=u, \gamma(1)=v$. $\V$ sends $\gamma(x)$ to $\P$. Then $\P$ sends $\V$ a degree $s_i$ univariate polynomial $h(x)=\tilde{V_i}(\gamma(x))$. $\V$ checks that $h(0)=\tV_i(u), h(1)=\tV_i(v)$. Then $\V$ randomly chooses $r\in\mathbb{F}$ and computes a new claim $h(r) = \tV_i(\gamma(r)) = \tV_i(w)$ on $w=\gamma(r) \in \mathbb{F}^{s_i}$. $\V$ sends $r, w$ to $\P$. In this way, the two claims are reduced to one claim $\tV_i(w)$. Combining this protocol with the sumcheck protocol on Equation~\ref{eq:GKR}, $\V$ and $\P$ can reduce a claim on layer $i$ to one claim on layer $i+1$, and eventually to a claim on the input, which completes the GKR protocol.




\smallskip\noindent\textbf{Combining two claims: random linear combination.}  In~\cite{zksumcheck}, Chiesa et al. proposed an alternative approach using random linear combinations. Upon receiving the two claims $\tV_i(u)$ and $\tV_i(v)$, $\V$ selects $\alpha_i, \beta_i\in\mathbb{F}$ randomly and computes $\alpha_i\tV_i(u)+\beta_i\tV_i(v)$. Based on Equation~\ref{eq:GKR}, this random linear combination can be written as
{\footnotesize
\begin{align}\label{eq:randomGKR}
&\alpha_i\tV_i(u)+\beta_i\tV_i(v)\nonumber\\
=&\alpha_i\sum_{x, y \in\binary^{s_{i+1}}}(\tadd_{i+1}(u,x,y)(\tV_{i+1}(x)+\tV_{i+1}(y))+\tmult_{i+1}(u,x,y)(\tV_{i+1}(x)\tV_{i+1}(y)))\nonumber\\+&\beta_i\sum_{x, y \in\binary^{s_{i+1}}}(\tadd_{i+1}(v,x,y)(\tV_{i+1}(x)+\tV_{i+1}(y))+\tmult_{i+1}(v,x,y)(\tV_{i+1}(x)\tV_{i+1}(y)))\nonumber\\
=&\sum_{x, y \in\binary^{s_{i+1}}}((\alpha_i\tadd_{i+1}(u,x,y)+\beta_i\tadd_{i+1}(v,x,y))(\tV_{i+1}(x)+\tV_{i+1}(y))\nonumber\\
&+(\alpha_i\tmult_{i+1}(u,x,y)+\beta_i\tmult_{i+1}(v,x,y))(\tV_{i+1}(x)\tV_{i+1}(y)))
\end{align}
}
$\V$ and $\P$ then execute the sumcheck protocol on Equation~\ref{eq:randomGKR} instead of Equation~\ref{eq:GKR}. At the end of the sumcheck protocol, $\V$ still receives two claims about $\tV_{i+1}$, computes their random linear combination and proceeds to an layer above recursively until the input layer.

In our new ZKP scheme, we will mainly use the second approach. The full GKR protocol using random linear combinations is given in Protocol~\ref{protocol::GKR} of Appendix~\ref{app:gkr}.

\begin{theorem}\cite{VSA13}\cite{JT_Thesis}\cite{CMT}\cite{GKR}. Let $C$ : $\mathbb{F}^n \rightarrow \mathbb{F}^k$ be a depth-$d$ layered arithmetic circuit. Protocol \ref{protocol::GKR} is an interactive proof for the function computed by $C$ with soundness $O(\frac{d\log {|C|}}{|\mathbb{F}|})$. It uses $O(d \log |C|)$ rounds of interaction and running time of the prover $\mathcal{P}$ is $O(|C|\log |C|)$. Let the optimal computation time for all $\tilde{add_i}$ and $\tilde{mult_i}$ be $T$, the running time of $\V$ is $O(n+k+d\log |C|+T)$. For log-space uniform circuits, $T=\poly{\log |C|}$.
\end{theorem}
%!TEX root = fastZKP.tex


