%!TEX root = fastZKP.tex

\subsection{CMT Protocol}
\label{subsec::CMT}
\subsubsection{Notation}
We define the polynomial for each layer as follows:
\begin{definition} [Polynomial for the circuit]
	$$V_i(g)=\sum_{u, v \in \{0,1\}^{s_{i+1}}}(add_{i+1}(g,u,v)(V_{i+1}(u)+V_{i+1}(v))+mult_{i+1}(g,u,v)(V_{i+1}(u)V_{i+1}(v)))$$
	where the gates are numbered by the binary label, $2^{s_i}$ is the number of gates in $i$-th layer, and $V_i(g)$ means the output of gate $g$ of $i$-th layer.
\end{definition}

We futher define the multi-linear extension of this polynomial. Note $G_{i,add}$ denote all add gate triples in $i$-th layer and so does $G_{i,mult}$

\begin{definition}[Multilinear extension of identity fucntion]
Here we present a multilinear extension of identity function
\begin{eqnarray}\beta_{l}(a, b)=
	\begin{cases}
	1, &a=b\cr 0, &a\neq b
	\end{cases}
\end{eqnarray}
Where $a, b$ are binary strings with length $l$. The multilinear extension is the following:

$$\tilde{\beta_{l}}(a,b)\overset{def}{=}\prod_{j=1}^{l}((1-a_{j})(1-b_{j})+a_{j}b_{j})$$

We have $\tilde{\beta_{l}}(a,b)=\beta_{l}(a, b)$ when $a, b$ are binary.
\end{definition}

\begin{definition}[Multilinear extension of $add$, $mult$]
We will define the multilinear extension of two component of $V_i$.
$$\tilde{add}_{i}(g, u, v)=\sum_{(g', u', v') \in G_{i, add}}\tilde{\beta}_{s_i}(g, g')\tilde{\beta}_{s_{i+1}}(u, u')\tilde{\beta}_{s_{i+1}}(v, v')$$

$$\tilde{mult}_{i}(g, u, v)=\sum_{(g', u', v') \in G_{i, mult}}\tilde{\beta}_{s_i}(g, g')\tilde{\beta}_{s_{i+1}}(u, u')\tilde{\beta}_{s_{i+1}}(v, v')$$
\end{definition}

\begin{definition}[Multilinear extension of $V_i(g)$]
	\label{def::multilinear}
	%$$\tilde{V}_{i}(z)=\sum_{g\in\{0,1\}^{s_i} u, v\in \{0,1\}^{s_{i+1}}}f_{i,z}(g,u,v)$$

	%$$f_{i,z}(g,u,v)\overset{def}{=}\tilde{\beta_{i}}(z, g)[\tilde{mult}(g, u, v)(\tilde{V}_{i+1}(u)\tilde{V}_{i+1}(v))+\tilde{add}(g,u,v)(\tilde{V}_{i+1}(u)+\tilde{V}_{i+1}(v))]$$
	%where $$\tilde{\beta_{i}}(z,g)\overset{def}{=}\prod_{j=1}^{s_{i}}((1-g_{j})(1-z_{j})+g_{j}z_{j})$$

	%$$\tilde{mult}(g,u,v)\overset{def}{=}\sum_{(g', u', v')\in G_{i,mult}}\tilde{\beta_{i}}(g,g')\tilde{\beta_{i+1}}(u,u')\tilde{\beta_{i+1}}(v,v')$$

	%$$\tilde{add}(g,u,v)\overset{def}{=}\sum_{(g', u', v')\in G_{i,add}}\tilde{\beta_{i}}(g,g')\tilde{\beta_{i+1}}(u,u')\tilde{\beta_{i+1}}(v,v')$$

	%when $a, b$ are binary strings, $\tilde{\beta_{i}}(a, b)$ will output $1$ when $a=b$, and $0$ otherwise. $\tilde{mult},\tilde{add}$ is the multilinear extension of the wiring predicates. A wiring predicates is a function of three gates $g_1, g_2, g_3$, and returns $1$ if $g_1$ takes $g_2, g_3$ as it's input.

	We define the multilinear extension of $V_i(g)$ in the following way:

	$$\tilde{V}_i(g)=\sum_{u, v \in \{0,1\}^{s_{i+1}}}(\tilde{add}_{i+1}(g, u, v)(\tilde{V}_{i+1}(u)+\tilde{V}_{i+1}(v))+\tilde{mult}_{i+1}(g,u,v)(\tilde{V}_{i+1}(u)\tilde{V}_{i+1}(v)))\textnormal{,}$$
	where $\tilde{mult}$ and $\tilde{add}$ are multilinear extension of $mult$ and $add$.
\end{definition}

\subsubsection{High Level Ideas}
CMT Protocol (Cormode et al.)\cite{CMT} is based on the work of Goldwasser et al. \cite{GKR}, gives us a efficient implementation of interactive proof. In this section, we will introduce the original CMT protocol here.

Assume $C$ is a \textit{layered arithmetic circuit} with depth $d$ over a finite field $\mathbb{F}$. The gates in $i$-th layer takes input from $i+1$-th layer and outputs to $i-1$-th layer; layer $0$ is the output layer, and layer $d$ is the input layer. 

%The protocol proceeds layer by layer. The prover starts by sending the output value to the verifier. Then the protocol starts from the output layer. In $i$-th round both party runs the sum check protocol on the previous prover's claim, for the first round, they run check sumcheck to check the output. The sumcheck will eventually reduce the claim into the evalution to one specific random point. Each point of current layer $f_{i,z}(R_g,R_u,R_v)$, corresponds to two points of next layer $V_{i+1}(R_u), V_{i+1}(R_v)$. A simple idea is to recursively check these two points. However the number of threads will grow exponentially, which would result in a very inefficient protocol.

%Instead of directly check two points, we can combine these two points into one by linear combination. And recursively check on the combined point of next layer. For the final input layer, the verifier checks the claim by calculating the value by himself.

The protocol proceeds layer by layer. The protocol start by checking the output layer ($0$-th layer). In the initialization, the verifier request a random claim about the output layer from the prover, and the prover answer the request by scanning through the layer. In the first round, both parties runs the sumcheck protocol on the claim about the output, and reduces to two claims about the next layer ($1$-st layer). The verifier will combine these two claims into one claim about $1$-st layer and runs into next round. In $i$-th round, both parties runs a sumcheck on the previous layer and reduce to a claim about current layer. Finally the protocol will get a claim about the input layer. The verifier will verify the claim by asking the oracle about the input layer. We will provide a implemetation about the oracle latter.

\subsubsection{Protocol Details}

The CMT protocol do sumchecks on polynomial $V_{i}$ for each layer and reduces the claim to next layer polynomial $V_{i+1}$. We will explain the protocol in detail.

\begin{protocol}
	\label{protocol::CMT}
	Let $\mathbb{F}$ be a prime field. Let $C$: $\mathbb{F}^n\rightarrow \mathbb{F}^k$ be a $d$-depth layered arithmetic circuit. $\mathcal{P}$ is the prover and $\mathcal{V}$ is the prover. $\mathcal{P}$ wants to convince that $y=C(x)$ where $x$ is the input from the verifier $\mathcal{V}$, and $y$ is the output array, the detailed protocol:

	\begin{itemize}
		\item Suppose without loss of generality, $|y|=2^k$, we can pad element if the actual length of $y$ is not a power of $2$.
		\item By the definition of $V_i(j)$, $V_0(j)$ is $y_j$, $j$-th element of the output array. The verifier chooses uniform $r_0 \in \mathbb{F}^k$. Both parties set $a_0=\tilde{V_0}(r_0)$
		\item For $i=1,...,d$:
		\begin{itemize}
			\item Both parties runs sumcheck protocol for $a_{i-1}$ on polynomial $\tilde{V_{i-1}}$, in the last step of sumcheck, verifier need to access the oracle to compute $$\tilde{add}_{i-1}(g, R_{u}, R_{v})(\tilde{V}_{i}(R_{u})+\tilde{V}_{i}(R_v))+\tilde{mult}_{i-1}(g, R_{u}, R_{v})(\tilde{V}_{i}(R_{u})\tilde{V}_{i}(R_v))\textnormal{,}$$ 
			where $R_{u}, R_{v}$ are chosen randomlly from $\mathbb{F}^{|u|},\mathbb{F}^{|v|}$. 

			\item Here we implement the oracle: the verifier asks the value of $\tilde{V}_i(u), \tilde{V}_i(v)$ from the prover, and the verifier compute $\tilde{add}, \tilde{mult}$ locally. 
			\item Then the verifier combines $\tilde{V_{i}}(R_{u}), \tilde{V_{i}}(R_{v})$ in this way: 
			\begin{itemize}
				\item Let $\gamma: \mathbb{F} \rightarrow \mathbb{F}^{s_i}$ be a line with $\gamma(0)=R_{u}, \gamma(1)=R_{v}$. 
				\item The verifier sends $\gamma(x)$ to the prover. 
				\item Then prover send a degree $s_i$ polynomial $h(x)=\tilde{V_{i}}(\gamma(x))$. The verifier verifies that $h(0)=v_0, h(1)=v_1$. And it compute $a_i=h(r_i)$ by randomly choose $r_i'$ and set $r_i=\gamma(r_i')$ and sends $r_i, a_i$ to prover.
			\end{itemize}
		\end{itemize}
		\item The verifier accepts iff $a_d = \tilde{V_d}(r_d)$, where $\tilde{V_d}$ is the multi-linear extension representation of input layer.
	\end{itemize}
\end{protocol}

\begin{theorem}\cite{VSA13}\cite{JT_Thesis}\cite{CMT}\cite{GKR}. Let $C$ : $\mathbb{F}^n \rightarrow \mathbb{F}^k$ be a depth-$d$ layered arithmetic circuit. Protocol \ref{protocol::CMT} is an interactive proof for the function computed by $C$ with soundness $O(\frac{d\log {S}}{|\mathbb{F}|})$, where $S$ is the maximal number of gates per circuit layer. It uses $O(d \log S)$ rounds of interaction and running time of prover $\mathcal{P}$ is $O(|C|\log S)$. Let the optimal computation time for $\tilde{add_i}$ and $\tilde{mult_i}$ be $T$, the running time of the verifier is $O(n+k+dT)$, in some applications, $T=\poly{\log S}$.
\end{theorem}