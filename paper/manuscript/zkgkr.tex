%!TEX root = fastZKP.tex

\section{Zero Knowledge Argument Protocols}\label{sec:zkp}

In this section, we present the construction of our new zero-knowledge argument system. In~\cite{zhang2017vsql}, Zhang et al. proposed to combine the GKR protocol with a verifiable polynomial delegation protocol, resulting in an argument system. Later, in~\cite{zkvpd,wahby2018doubly}, the construction was extended to zero-knowledge, by sending all the messages in the GKR protocol in homomorphic commitments and performing all the checks by zero-knowledge equality and product testing. This incurs a high overhead for the verifier compared to the plain version without zero-knowledge, as each multiplication becomes a homomorphic operation (exponentiation) and each equality check becomes a $\Sigma$-protocol, which is around $100\times$ slower in practice.

In this paper, we follow the same blueprint of combining GKR and VPD to obtain an argument system, but instead show how to extend it to be zero-knowledge efficiently. In particular, the prover masks the GKR protocol with special random polynomials so that the verifier runs a "randomized" GKR that leaks no extra information and her overhead is small. A similar approach was used by Chiesa et.al in~\cite{zksumcheck}. In the following, we present the zero-knowledge version of each building block, followed by the whole zero-knowledge argument protocol.

\subsection{Zero Knowledge Sumcheck}\label{subsec:zksumcheck}
As a core step of the GKR protocol, $\P$ and $\V$ execute a sumcheck protocol on Equation~\ref{eq:GKR}, during which $\P$ sends $\V$ evaluations of the polynomial at several random points chosen by $\V$. These evaluations leak information about the values in the circuit, as they can be viewed as weighted sums on the values in each layer of the circuit. 

To make the sumcheck protocol zero-knowledge, we take the approach proposed by Chiesa et al. in \cite{zksumcheck}, which is masking the polynomial in the sumcheck protocol by a random polynomial. In this approach, to prove $H = \sum\limits_{x_1, x_2, \cdots, x_\ell \in \binary}f(x_1, x_2, \cdots, x_\ell)$, the prover generates a random polynomial $g$ with the same variables and individual degrees of $f$. She commits to the polynomial $g$, and sends the verifier a claim $G = \sum\limits_{x_1, x_2, \cdots, x_\ell \in \binary}g(x_1, x_2, \cdots, x_\ell)$. The verifier picks a random number $\rho$, and execute a sumcheck protocol with the prover on $$H + \rho G = \sum\limits_{x_1, x_2, \cdots, x_\ell \in \{0, 1\}}(f(x_1, x_2, \cdots, x_\ell) + \rho g(x_1, x_2, \cdots, x_\ell)).$$ At the last round of this sumcheck, the prover opens the commitment of $g$ at $g(r_1, \ldots, r_\ell)$, and the verifier computes $f(r_1, \ldots, r_l)$ by subtracting $\rho g(r_1, \ldots, r_\ell)$ from the last message, and compares it with the oracle access of $f$. It is shown that as long as the commitment and opening of $f$ are zero-knowledge, the protocol is zero-knowledge. Intuitively, this is because all the coefficients of $f$ are masked by those of $g$. The soundness still holds because of the random linear combination of $f$ and $g$. 

Unfortunately, the masking polynomial $g$ is as big as $f$, and opening it to a random point later is expensive. In~\cite{zksumcheck}, the prover sends a PCP oracle of $g$, and executes a zero-knowledge sumcheck to open it to a random point, which incurs an exponential complexity for the prover. Even replacing it with the zkVPD protocolin~\cite{zkvpd}, the prover time is slow in practice.

In this paper, we show that it suffices to mask $f$ with a small polynomial to achieve zero-knowledge. In particular, we set $g(x_1, \ldots, x_\ell) = a_{0} + g_1(x_1) + g_2(x_2) + \ldots + g_\ell(x_\ell)$, where $g_{i}(x_i) = a_{i,1}x_i + a_{i,2}x_i^2 + \ldots + a_{i,d}x_i^d$ is a random univariate polynomial of degree $d$ ($d$ is the variable degree of $f$). Note here that the size of $g$ is only $O(d\ell)$, while the size of $f$ is exponential in $\ell$ ($O(2^\ell)$ in the GKR protocol).

The intuition of our improvement is that the prover sends $O(d\ell)$ messages in total to the verifier during the sumcheck protocol, thus a polynomial $g$ with $O(d\ell)$ random coefficients is sufficient to mask all the messages and achieve zero-knowledge. We present the full protocol below. \yupeng{find a protocol package}


\medskip\noindent\textbf{Protocol.}

We assume the existence of a zkVPD protocol defined in Section~\ref{subsec::zkvpd}. For simplicity, we omit the randomness $r_f$ and public parameters $\pp,\vp$ without any ambiguity. To prove the claim $H = \sum\limits_{x_1, x_2, \cdots, x_\ell \in \binary} f(x_1, x_2, \cdots, x_\ell)$:
\begin{enumerate}

\item $\P$ selects a polynomial $g(x_1,\ldots, x_\ell) = a_{0} + g_1(x_1) + g_2(x_2) + \cdots + g_l(x_\ell)$, where $g_{i}(x_i) = a_{i,1}x_i + a_{i,2}x_i^2 + \cdots + a_{i,d}x_i^d$ and all $a_{i,j}$s are uniformly random. $\P$ sends $H = \sum\limits_{x_1, x_2, \cdots, x_\ell \in \binary} f(x_1, x_2, \cdots, x_\ell)$, $G = \sum\limits_{x_1, x_2, \cdots, x_\ell \in \binary} g(x_1, x_2, \cdots, x_\ell)$ and $\comm_g = \Commit(g)$ to $\V$.
\item $\V$ uniformly selects $\rho \in \mathbb{F}^*$, computes $H+\rho G$ and sends $\rho$ to $\P$.
\item $\P$ and $\V$ run the sumcheck protocol on
$$H + \rho G = \sum\limits_{x_1, x_2, \cdots, x_\ell \in \binary}(f(x_1, x_2, \cdots, x_\ell) + \rho g(x_1, x_2, \cdots, x_\ell))$$
\item At the last round of the sumcheck protocol, $\V$ obtains a claim $h_\ell(r_\ell) = f(r_1, r_2, \cdots, r_\ell)+\rho g(r_1, r_2, \cdots, r_\ell)$. $\P$ and $\V$ opens the commitment of $g$ at $r = (r_1,\ldots, r_\ell)$ by $(g(r), \pi)\leftarrow\Open(g,r), \Verify(\comm_g,g(r),r,\pi)$. If $\Verify$ outputs $\reject$, $\V$ aborts.
\item $\V$ computes $h_\ell(r_\ell)-\rho g(r_1,\ldots,r_\ell)$ and compares it with the oracle access of $f(r_1,\ldots, r_\ell)$.

\end{enumerate}

The completeness of the protocol holds obviously. The soundness of the protocol follows the soundness of the sumcheck protocol and the random linear combination in step 2 and 3, as proven in~\cite{zksumcheck}. We give a proof of zero-knowledge here.

\begin{theorem}\label{thm:zksc}
	For every verifier $\V^*$ and every $\ell$-variate polynomial $f:\F^\ell\rightarrow\F$ with variable degree $d$, there exists a simulator $\S$ such that given access to $H = \sum\limits_{x_1, x_2, \cdots, x_\ell \in \binary} f(x_1, x_2, \cdots, x_\ell)$, $\S$ is able to simulate the partial view of $\V^*$ in step 1-4 of Protocol~\ref{??}. 
\end{theorem}

\begin{proof}
	
	We build the simulator $\S$ as following.
	
	\begin{enumerate}
		
		\item $\S$ selects a random polynomial $g^*(x_1,\ldots,x_\ell) = a^*_{0} + g^*_1(x_1) + g^*_2(x_2) + \cdots + g^*_\ell(x_\ell)$, where $g^*_i(x_i) = a^*_{i,1}x_i + a^*_{i,2}x_i^2 + \cdots + a^*_{i,d}x_i^d$. $\S$ sends $H$, $G^* = \sum\limits_{x_1, x_2, \cdots, x_\ell \in \binary} g^*(x_1, x_2, \cdots, x_\ell)$ and $\comm_{g^*} = \Commit(g^*)$ to $\V$.
		
			
		\item $\S$ receives $\rho \neq 0$ from $\V^*$.
		\item $\S$ selects a polynomial $f^*:\F^\ell\rightarrow\F$ with variable degree $d$ uniformly at random conditioning on $\sum\limits_{x_1, x_2, \cdots, x_\ell \in \{0, 1\}}f^*(x_1, x_2, \cdots, x_\ell) = H$. $\S$ then engages in a sumcheck protocol with $\V$ on  $H+\rho G^* = \sum\limits_{x_1, x_2, \cdots, x_l \in \{0, 1\}}(f^*(x_1, x_2, \cdots, x_\ell)+\rho g^*(x_1, x_2, \cdots, x_\ell))$
		
		\item Let $r \in \F^\ell$ be the point chosen by $\V^*$ in the sumcheck protocol. $\S$ runs $(g^*(r), \pi)\leftarrow\Open(g^*,r)$ and sends them to $\V$.
		
	\end{enumerate} 
	
	As both $g$ and $g^*$ are randomly selected, and the zkVPD protocol is zero-knowledge, it is obvious that step 1 and 4 in $\S$ are indistinguishable from those in the real world of Protocol~\ref{??}. It remains to show that the sumchecks in step 3 of both worlds are indistinguishable.
	
	To see that, recall that in round $i$ of the sumcheck protocol, $\V$ receives a univariate polynomial $h_i(x_i) = \sum\limits_{b_{i+1}, \ldots,b_\ell\in\binary}h(r_1, \ldots, r_{i-1},x_i, b_{i+1}, \ldots, b_\ell)$ where $h = f+\rho g$. (The view of $\V^*$ is defined in the same way with $h^*,f^*,g^*$ and we omit the repetition in the following.) As the variable degree of $f$ and $g$ is $d$, $\P$ sends $\V$ $h_i(0), h_i(1), \ldots, h_i(d)$ which uniquely defines $h_i(x_i)$. These evaluations reveal $d+1$ independent linear constraints on the coefficients of $h$. In addition, note that when these evaluations are computed honestly by $\P$, $h_i(0)+h_i(1) = h_{i-1}(r_{i-1})$, as required in the sumcheck protocol. Therefore, in all $\ell$ rounds of the sumcheck, $\V$ and $\V^*$ receives $\ell(d+1) - (\ell-1) = \ell d+1$ independent linear constraints on the coefficients of $h$ and $h^*$. 
	
	As $h$ and $h^*$ are masked by $g$ and $g^*$, each with exactly $\ell d+1$ coefficients selected randomly, the two linear systems are identically distributed. Therefore, step 3 of the ideal world is indistinguishable from that of the real world.
	
\end{proof}



\subsection{Zero knowledge GKR}

To achieve zero-knowledge, we replace the sumcheck protocol in GKR with the zero-knowledge version described in the previous section. However, the protocol still leaks additional information. In particular, at the end of the zero-knowledge sumcheck, $\V$ queries the oracle to evaluate the polynomial on a random point. When executed on Equation~\ref{eq:GKR}, this reveals two evaluations of the polynomial $\tV_i$ defined by the values in the $i$-th layer of the circuit: $\tV_i(u)$ and $\tV_i(v)$.


To prevent this leakage, Chiesa et al.\cite{zksumcheck} proposed to replace the multi-linear extension $\tV_i$ with a low degree extension, such that learning $\tV_i(u)$ and $\tV_i(v)$ does not leak any information about $V_i$. In particular, define a low degree extension of $V_i$ as 

\begin{equation}\label{eq:lde}
\dot{V}_{i}(z) \overset{def}{=} \tV_i(z)+Z_i(z)\sum\limits_{w \in \{0, 1\}^\lambda}R_i(z, w),
\end{equation}

where $Z(z) = \prod_{i=1}^{s_i} z_i(1-z_i)$, i.e., $Z(z)=0$ for all $z\in\{0, 1\}^{s_i}$. $R_i(z,w)$ is a random low-degree polynomial and $\lambda$ is the security parameter. 

With this low degree extension, Equation~\ref{eq:GKR} becomes

\begin{align}\label{eq:zkGKR}
\dot{V}_{i}(g)=\sum_{x, y\in \binary^{s_{i+1}}}\tilde{mult}_{i+1}(g, x, y)(\dot{V}_{i+1}(x)\dot{V}_{i+1}(y))&+\tilde{add}_{i+1}(g,x,y)(\dot{V}_{i+1}(x)+\dot{V}_{i+1}(y))\nonumber\\
&+ Z_i(g)\sum\limits_{w \in \binary^\lambda}R_i(g, w)\nonumber\\
=\sum_{x, y\in \binary^{s_{i+1}},w \in \binary^\lambda}(I(\vec{0},w) \cdot \tilde{mult}_{i+1}(g, x, y)(\dot{V}_{i+1}(x)\dot{V}_{i+1}(y))&+\tilde{add}_{i+1}(g,x,y)(\dot{V}_{i+1}(x)+\dot{V}_{i+1}(y))\nonumber\\
&+ I((x, y), \vec{0})Z_i(g)R_i(g, w))
\end{align}
where $I(\vec{a},\vec{b})$ is an identity polynomial $I(\vec{a},\vec{b}) = 0$ iff $\vec{a}=\vec{b}$. \yupeng{explain why it holds.}


When executing the zero-knowledge sumcheck protocol on Equation~\ref{eq:zkGKR}, at the end of the protocol, $\V$ receives $\dot{V}_{i+1}(u)$ and $\dot{V}_{i+1}(v)$ for random points $u,v\in\F^{s_{i+1}}$ chosen by $\V$. They no longer leak information about $V_{i+1}$, as it is masked by $Z_{i+1}(z)\sum\limits_{w \in \{0, 1\}^\lambda}R_{i+1}(z, w)$ for $z=u$ and $z=v$. $\V$ computes $\tilde{mult}_{i+1}(g,u,v)$ and $\tilde{add}_{i+1}(g,u,v)$ as before, computes $Z_i(g), I(\vec{0},c), I((u,v),\vec{0})$ where $c\in\F^\lambda$ is a random point chosen by $\V$ for variable $w$, opens $R_i(g,w)$ at $c$  with $\P$ through a polynomial commitment, and checks that together with $\dot{V}_{i+1}(u), \dot{V}_{i+1}(v)$ received from $\P$ they are consistent with the last message of the sumcheck.$\V$ then uses $\dot{V}_{i+1}(u), \dot{V}_{i+1}(v)$ to proceed to the next round.

Unfortunately, similar to the zero-knowledge sumcheck, the masking polynomail $R_i$ is very large in~\cite{zksumcheck}. Opening $R_i$ at a random point takes exponential time for $\P$ either using a PCP oracle as in~\cite{zksumcheck} or potentially using a zkVPD, as $R$ has $s_i+2s_{i+1}+\lambda$ variables.

In this section, we show that we can set $R_i$ to be a small polynomial to achieve zero-knowledge. In particular, $R_i$ has only two variables with variable degree 2. This is because in the $(i-1)$-th round, $\V$ receives two evaluations of $V_i$, $\dot{V}_i(u)$ and $\dot{V}_i(v)$,  which are masked by $\sum_{w}R_i(u,w)$ and $\sum_{w}R_i(v,w)$; in the $i$-th sumcheck, $\V$ opens $R_i$ at $\alpha R_i(u,c)+\beta R_i(v,c)$. It suffices to make these three evaluations linearly independent, assuming the commitment and opening of $R_i$ are using a zkVPD. Therefore, we set the low-degree term in Equation~\ref{eq:lde} as $Z_i(z)\sum\limits_{w \in\binary} R_i(z_1, w)$, i.e. $R_i$ only takes two variables: the first variable $z_1$ of $z$ and the length of $w\in\binary$ instead of $\binary^\lambda$. We present the protocol in detail below.



\medskip\noindent\textbf{Protocol.}
\begin{enumerate} 
\item The prover $\mathcal{P}$ gives $\text{Output}$ of the circuit $\dot{V}_0(0)$ to the verifier $\mathcal{V}$. 
\item $\mathcal{P}$ samples $R_1(X, Y), R_2(X, Y), \cdots, R_d(X, Y)$ randomly from $\mathbb{F}[X^{\leqslant 2}, Y^{\leqslant 2}]$
\item $\mathcal{P}$ and $\mathcal{V}$ run the zero knowledge sumcheck protocol on the claim:
$$\dot{V}_0(0) = \sum_{u, v\in \{0,1\}^{s_1}}\tilde{mult}(0, u, v)(\dot{V}_1(u)\dot{V}_1(v))+\tilde{add}(0,u,v)(\dot{V}_1(u)+\dot{V}_1(v))$$

At the end of the sumcheck protocol, $\mathcal{P}$ sends $\dot{V}_1(r)$ and $\dot{V}_1(s)$ to $\mathcal{V}$ if $r_1 \neq s_1$, where $r, s$ are randomly chosen by $\mathcal{V}$, If $r_1 = s_1$ $\mathcal{P}$ aborts. If it does not pass the sum check protocol, $\mathcal{V}$ rejects. 

\item For layer $i(1 \leq i \leq d - 1)$:
	\begin{itemize}
	\item $\mathcal{V}$ samples $\alpha_1, \alpha_2$ at random and sends them to $\mathcal{P}$.
	\item $\mathcal{V}$ and $\mathcal{P}$ run the zero knowledge sumcheck on the claim
	$$\alpha_1 \dot{V}_i(r) + \alpha_2 \dot{V}_i(s) = \alpha_1 a_{r} + \alpha_2 a_{s}$$
	Let $z^{(1)} = r$ and $z^{(2)} = s$.
	\item At the end of the sumcheck protocol, $\mathcal{P}$ receives the random vector $r, s, p$ with $r, s \in (\mathbb{F} - \{0, 1\})^{s_{i+1}}$ and $p \in \mathbb{F}$. 
	\begin{align*}
		\sum_{j = 1, 2} \alpha_j (\tilde{mult}(z^{(j)}, r, s)(\dot{V}_{i+1}(r)\dot{V}_{i+1}(s))&+\tilde{add}(z^{(j)},r,s)(\dot{V}_{i+1}(r)+\dot{V}_{i+1}(s))\\
 		&+ Z_i(z^{(j)})\sum\limits_{c \in \{0, 1\}}R_i(z^{(j)}_1, c))
	\end{align*}
	\item If $r_1 = s_1$, abort. 
	\item $\mathcal{P}$ replies with the value $a_r = \dot{V}_{i+1}(r)$, $a_s = \dot{V}_{i+1}(s)$.
	\item $\mathcal{V}$ queries the oracle to get $a = R_i(z^{(1)}_1, p)$ and $b = R_i(z^{(2)}_1, p)$. If $\mathcal{P}$ does not pass the sumcheck protocol, $\mathcal{V}$ rejects.
	\end{itemize}

\item For input layer $d$, $\mathcal{V}$ queries the oracle to verify the claim $\dot{V}_d(r) = \tilde{V}_d(r) + Z(r)\sum\limits_{c \in \{0, 1\}}R_d(r_1, c)$ and $\dot{V}_d(s) = \tilde{V}_d(s) + Z(s)\sum\limits_{c \in \{0, 1\}}R_d(s_1, c)$. $\mathcal{V}$ accepts if it passes the final verification. Otherwise, $\mathcal{V}$ rejects. 
\end{enumerate}

\noindent
\textbf{Completeness}. The completeness property immediately follows from the construction of the protocol and the completeness of zero knowledge sumcheck. We now proceed to argue about the soundness.\\

\noindent
\textbf{Soundness}. Let $\mathcal{P}^*$ be an cheating prover which convinces $\mathcal{V}$ of a claim ``Output = C(Input)'' for some Input and C such that Output $\neq$ C(Input). That means $\dot{V_0(0)} \neq \text{Output}$ in the beginning of the protocol. Supoose that we omit the soundness of the oracle since all of them are negligible of $\lambda = \log |\mathbb{F}|$. We claim that if there exists $\dot{V}_i(r) \neq a_r$, then after this iteration with high probability either $(a)$ $\mathcal{V}$ rejects, or $(b)$ in the next iteration, $\dot{V}_{i+1}(r) \neq a_r$ or $\dot{V}_{i+1}(s) \neq a_s$. Since $\alpha_1$ and $\alpha_2$ are randomly chosen by $\mathcal{V}$, the probability that $\alpha_1 \dot{V}_i(r) + \alpha_2 \dot{V}_i(s) = \alpha_1 a_{r} + \alpha_2 a_{s}$ is $1/|\mathbb{F}|$ if $\dot{V}_i(r) \neq a_r$. Besides, according to the soundness of sumcheck protobol, when we run the protocol on a false claim with $2s_{i+1} + 1 \leq 3 \log S$ variables and individual degree at most 2, the verifier $\mathcal{V}$ either rejects or passses the false claim to the next iteration with probability $1 - 6 \log S / (|\mathbb{F}| - 2)$.\\

That is to say, if $C(\text{Input}) \neq \dot{V}_0(0)$, the verifier $\mathcal{V}$ will accept with probability at most $\mathcal{O}(d \log S / |\mathcal{F}|)$, where $d$ is the depth of $C$ and $S$ is the maximum number of gates in one layer of $C$. Hence, the total soundness of the zero knowledge GKR protocol is negligible of $\lambda$.\\

\noindent
\textbf{Zero knowledge}. We prove that the protocol has perfect zero knowledge by exhibiting a polynomial-time simulator $\mathcal{S}$ with $\text{Ouptut}$ and the circuit $C$ that perfectly samples the view of any malicious verifier. 

\begin{enumerate}

\item For every layer $i(1 \leq i \leq d)$ and the function $\dot{V}_i(z)$, sample $R_i^{sim} \in \mathbb{F}[X^{\leqslant 2}, Y^{\leqslant 2}]$ uniformly at random satisfying some conditions. Use $R_i^{sim}$ to answer queries to $R_i$.

\item For every layer function $\dot{V}_i(z)$, run the zero knowledge sumcheck subsimulator for the sumcheck of $\dot{V}_i(z)$, and use it to answer quries of the verifier $\mathcal{V}^*$. 

\item For the output layer(layer $0$), run the zero knowledge sumcheck subsimulator on the claim

$$\text{Output} = \dot{V}_0(0) = \sum_{u, v\in \{0,1\}^{s_1}}\tilde{mult}(0, u, v)(\dot{V}_1(u)\dot{V}_1(v))+\tilde{add}(0,u,v)(\dot{V}_1(u)+\dot{V}_1(v))$$ 
At the end of the zero knowledge sumcheck protocol, the simulator receives the random vector $q = (r, s)$ with with $r, s \in \mathbb{F}^{s_1}$. If $r_1 = s_1$, abort. Otherwise, sample $\dot{V}_1(r)$ and $\dot{V}_1(s)$ at random conditioned on satifying the sumcheck protocol. 
\item For layer $i(1 \leq i \leq d - 1)$:
	\begin{itemize}
	\item Receive $\alpha_1$ and $\alpha_2$ from the verifier.
	\item Using the subsimulator for $\dot{V}_i$, simulate the strong zero knowledge sumcheck protocol on the claim
	$$\alpha_1 \dot{V}_i(r) + \alpha_2 \dot{V}_i(s) = \alpha_1 a_{r} + \alpha_2 a_{s}$$
	Let $z^{(1)} = r$ and $z^{(2)} = s$. 
	\item Then the simulator receives the random vector $q = (r, s, p)$ with $r, s \in (\mathbb{F}-\{0, 1\})^{s_{i+1}}$ and $p \in \mathbb{F}$. 
	\item If $r_1 = s_1$, abort. 
	\item Otherwise, reply with the value
	\begin{align}
		\sum_{j = 1, 2} \alpha_j (\tilde{mult}(z^{(j)}, r, s)(\dot{V}_{i+1}(r)\dot{V}_{i+1}(s))&+\tilde{add}(z^{(j)},r,s)(\dot{V}_{i+1}(r)+\dot{V}_{i+1}(s))\\
 		&+ Z_i(z^{(j)})\sum\limits_{c \in \{0, 1\}}R_i(z_1^{(j)}, c))
	\end{align}
	\item Sample $\dot{V}_{i+1}(r)$ and $\dot{V}_{i+1}(s)$ at random. And randomly choose the function $R^{sim}_i(X, Y)$ satisfying the above claim. The verifier queries the oracle to get $R^{sim}_i(z_1^{(1)}, p)$ and $R^{sim}_i(z_1^{(2)}, p)$.   
	\end{itemize}
\item For input layer $d$, the verifier queries the oracle to verify the values of $\dot{V}_d(r)$ and $\dot{V}_d(s)$.
\end{enumerate} 

If we want to make it zero knowledge, we need to prove that $\{View_{V^*}(\mathcal{P}(\text{Output, C, Input}) \leftrightarrow \mathcal{V^*}(\text{Output, C}))\} = \{\mathcal{S}^{\mathcal{V^*}}(\text{Output, C})\}$. We claim that in the real world execution, the distribution of $\dot{V}_i(r)$ and $\dot{V}_i(s)$ are also uniformly at random in $\mathbb{F}$. In the real world, for every mask function $R_i$, the verifier only knows the value of $R_i(r_1, p)$ and $R_i(s_1, p)$, where $r_1 \neq s_1$ and $r_1, s_1, p$ are randomly chosen by the verifier. If the verifier still have any no idea about the value of $\sum\limits_{c \in \{0, 1\}}R_i(r_1, c)$ and $\sum\limits_{c \in \{0, 1\}}R_i(s_1, c)$ conditioned on it, then $\dot{V}_i(r)$ and $\dot{V}_i(s)$ are uniformly at random.\\

Therefore, we only need to claim that $R(x, p), R(y, p), \sum\limits_{c \in \{0, 1\}}R(x, c), \sum\limits_{c \in \{0, 1\}}R(y, c)$ are independent if $R(x, y)$ are uniformly sampled from $\mathbb{F}[X^{\leqslant 2}, Y^{\leqslant 2}]$ and $x \neq y$. That is obvious since $x \neq y$ and both of the degrees of $X, Y$ are two.\\

Other things are also distinguishable for the verifier $\mathcal{V^*}$ because of the zero knowledge oracle(ZK VPD) and the zero knowledge sumcheck protocol. So our entire protocol is zero knowledge, which completes our proof.\\

\noindent
\textbf{Complexity}. Consider the time complexity of zero knowledge GKR protocol. We would like to focus on the prover time. Since we have analyzed the prover time in our linear GKR protocol, we only need to think about the overhead complexity while maintaining zero knowledge for GKR protocol. For every layer of the circuit $C$, the prover samples $R_i(X, Y)$, runs the zeor knowledge sumcheck protocol for at most $3s_i$ variables with individual degree at most $2$ and run the zero knowledge oracle(ZK VPD) for $R(X, Y)$ on $2$ random points. The total complexity is $\mathcal{O}(s_i)$ since the size of $R_i(X, Y)$ is only constant. Suppose the depth of the circuit $C$ is $d$ and the maximum number of gates in one layer is $S$. The total complexity is $\mathcal{O}(d \log S)$. In addition, at the end of the protocol, the prover needs to run the zero knowledge oracle(ZK VPD) for $\dot{V}_d$ on $2$ random points. It needs $\mathcal{O}(S)$. So the total complexity is $\mathcal{O}(S \times d)$, which is linear of the circuit size and it is optimal as the prover must evaluate the circuit $C$. 

\subsection{Zero knowledge VPD}\label{sec::zkvpd}
In this section, we explain how to exemplify the oracles using the zero knowledge VPD protocol. In the preliminary, we have defined the general zero knowledge VPD. We could use it as an oracle to verify any polynomial evaluations. However, at the end of our zero knowledge GKR protocol, for the input layer, the verifier needs to verify $\tilde{V}_d(r) + Z(r)\sum\limits_{c \in \{0, 1\}}R_d(r_1, c)$ and $\tilde{V}_d(s) + Z(s)\sum\limits_{c \in \{0, 1\}}R_d(s_1, c)$, which are not multi-linear polynomials. Here we show that we only need to add several extra terms in the public key of the original zero-knowledge VPD for multi-linear polynomials proposed in~\cite{zkvpd}. The overhead for commit, prove and verification are also minimal.

\medskip\noindent\textbf{Protocol.}
Let $\mathbb{F}$ be a prime-order finite field, $l$ be a variable parameter, and $d$ be a variable-degree parameter such that $\mathcal{O}(1)$ is polynomial in $\lambda$. For our problem, $l = s_n$ and $d = 1$.
\begin{enumerate}
\item \textbf{KeyGen($1^{\lambda}, l, d$):} Select $\alpha, \beta, t_1, t_2, \cdots, t_l, t_{l+1} \in \mathbb{F}$ uniformaly at random, run $bp \leftarrow \text{BilGen}(1^{\lambda})$ and compute $\mathbb{P} = \{g^{\Pi_{i \in W}t_i}, g^{\alpha \Pi_{i \in W}t_i}\}_{W \in \mathcal{W}_{l, d}} \cup \{g^{t_1}\}$. The public parameters are set to be $pp = (bp, \mathbb{P}, g^{\alpha}, g^{\beta}, g^{t_{l+1}}, g^{\alpha t_{l+1}}, g^{\beta t_{l+1}})$ and the verifier parameters are set to be $vp = (bp, g^{t_1}, g^{t_2}, \cdots, g^{t_{l+1}}, g^{\alpha}, g^{\beta})$.

\item \textbf{CommitPoly($\tilde{V}_d, p_1, pp$):} Compute $c_1 = g^{V_d(t_1, t_2, \cdots, t_l) + p_1t_{l+1}}$ and $c_2 = g^{\alpha(V_d(t_1, t_2, \cdots, t_l)+ p_1t_{l+1})}$, output the commitment $com_{R_d} = (c_1, c_2)$.

\item \textbf{CommitPoly($R_d, p_2, pp$):} Compute $c_3 = g^{R_d(t_1) + p_2t_{l+1}}$ and $c_4 = g^{\alpha(R_d(t_1)+ p_2t_{l+1})}$, output the commitment $com_{V_d} = (c_3, c_4)$.

\item \textbf{CommitPoly($\tilde{V}_d + Z(r)R_d, p_1 + Z(r)p_2, pp$):} Compute $c_5 = c_1 \cdot c_3^{Z(r)}$ and $c_6 = c_2 \cdot c_4^{Z(r)}$, output the commitment $com_{\tilde{V}_d + Z(r)R_d} = (c_5, c_6)$.

\item \textbf{CheckCom($com_{\tilde{V}_d}, com_{R_d}, com_{\tilde{V}_d + Z(r)R_d}$, vp):} Output accept if $e(com_{\tilde{V}_d, 1}, g^\alpha) = e(com_{\tilde{V}_d, 2}, g)$and $e(com_{R_d, 1}, g^\alpha) = e(com_{R_d, 2}, g)$ and $e(com_{\tilde{V}_d + Z(r)R_d, 1}, g^\alpha) = e(com_{\tilde{V}_d + Z(r)R_d, 2}, g)$, reject otherwise.

\item \textbf{CommitValue($\tilde{V}_d + Z(r)R_d, r, \tilde{V}_d(r) + Z(r)R_d(r_1), p_1 + Z(r)p_2, p_3, pp$):} the same as the \text{CommitValue} function in general ZK VPD protocol. 
\item \textbf{Ver($com_{V_d + Z(r)R_d}, r, com_{\tilde{V}_d(r) + Z(r)R_d(r_1)}, \pi, vp$):} the same as the \text{Ver} function in general ZK VPD protocol. 
\end{enumerate} 

Notice the verifier could verify the value of $\tilde{V}_d(r) + Z(r)R_d(r_1)$ and $\tilde{V}_d(s) + Z(s)R_d(s_1)$ using this protocol and it is still zero knowledge since the general ZK VPD protocol is zero knowledge. 

For the masking polynomials $G(x)$ and $R_i(x,y)$ in each layer, we use the original zero-knowledge VPD. As the masking polynomials are small, it is not hard to see that the public parameters used in those zk-VPDs are subsets of the public key presented above. The commitment time, prover time, proof size and verification time of each instance are all $O(s_i)$, logarithmic on the number of gates in each layer. 


\subsection{Putting Everything Together}

\yupeng{A figure for the whole scheme.}

\begin{theorem}\label{theorem:main}
	Construction~\ref{??} is a zero-knowledge argument system per Definition~\ref{??}, under Assumptions~\ref{??}. Keygen, prove, vtime, proof size.
\end{theorem}






