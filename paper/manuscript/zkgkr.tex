\section{Zero Knowledge Sumcheck}

In $GKR$ protocol,for every layer of the circuit, the prover needs to prove one equation to verifier based on the sumcheck protocol. That means if we want to get the zero knowledge $GKR$ protocol, we should first design an efficient zero knowlege sumcheck.\\

\noindent
For sumcheck protocol, the prover needs to prove $a = \sum\limits_{x_1, x_2, \cdots, x_n \in \{0, 1\}}f(x_1, x_2, \cdots, x_n)$ to the verifier. And the verifier accepts it if and only if the verifier verifies $a = f(r_1, r_2, \cdots, r_n)$, where $r_1, r_2, \cdots, r_n$ are randomly chosen by verifier and $a$ is given by the prover, at the end of the protocol. We could use verifiable polynomial delegation to help verifier verify the value of $f(r_1, r_2, \cdots, r_n)$. So Zero knowledge sumcheck means that the verifier could learn nothing about the polynomial $f$ except for the value of $a$.\\

\noindent
So the idea to make sumcheck protocol zero knowledge is very easy. We hope to use another sumcheck polynomial with the same degree and variables to mask $\sum\limits_{x_1, x_2, \cdots, x_n \in \{0, 1\}}f(x_1, x_2, \cdots, x_n)$. In addition, we hope that the mask polynomial is as simple as possible since the time complexity of verifiable polynomial delegation is related to the number of items in the polynomial.\\

\noindent
Consider the simplest example, where $f(x_1, x_2, \cdots, x_n)$ is a multilinear polynomial or the degree of $f$ is only one. Then we hope to use $g(x_1, x_2, \cdots, x_n) = a_0 + a_1x_1 + a_2x_2 + \cdots + a_nx_n$ to mask $f$, where $a_0, a_1, \cdots, a_n$ is randomly chosen from $\mathbb{F}$. Suppose that $b = \sum\limits_{x_1, x_2, \cdots, x_n \in \{0, 1\}}g(x_1, x_2, \cdots, x_n)$ and we would use the sumcheck protocol to verify

$$a + b = \sum\limits_{x_1, x_2, \cdots, x_n \in \{0, 1\}}[f(x_1, x_2, \cdots, x_n) + g(x_1, x_2, \cdots, x_n)]$$. 

In sumcheck protocol, we know that in the beginning, the prover sends $a + b$ to the verifier. Later on, the prover sends $g_i(x_i)$to the verifier in round $i$, where $1 \leq i \leq n$. And the verifier verifies $g_i(0) + g_i(1) = g_{i-1}(r_{i-1})$. Finally, the verifier queries $f(r_1, r_2, \cdots, r_n)$ and verifies whether it equals to $g_n(r_n)$.\\

\noindent
In our simple example, where $f$ and $g$ is multilinear, the prover could use $g_i(0)$ and $g_i(1)$ to replace $g_i(x_i)$. So we could construct our simulator $\mathcal{S}$. The simulator $\mathcal{S}$ given straightline access to $\mathcal{V^*}$ and the oracle access to $f$, works as follows

\begin{enumerate}

\item Draw a multilinear polynomial $Z_{sim} = a_0 + a_1x_1 + \cdots + a_nx_n$, where $a_0, a_1, \cdots, a_n$ are uniformly randomly chosen from $\mathbb{F}$. 

\item Draw a multilinear polynomial $Q_{sim} = \in \mathbb{F}[x_{1, 2, \cdots, n}^{\leq 1}]$ uniform at random conditioned on $a = \sum\limits_{x_1, x_2, \cdots, x_n \in \{0, 1\}}Z_{sim}(x_1, x_2, \cdots, x_n)$.

\item Begin simulating $\mathcal{V}^*$. The simulator sends $b = \sum\limits_{x_1, x_2, \cdots, x_n \in \{0, 1\}}Z_{sim}(x_1, x_2, \cdots, x_n)$ to $\mathcal{V}^*$. Then engage in the sumcheck protocol on the claim ``$\sum\limits_{x_1, x_2, \cdots, x_n \in \{0, 1\}}Q_{sim}(x_1, x_2, \cdots, x_n) + Z_{sim}(x_1, x_2, \cdots, x_n) = a + b$''. 

\item Let $\vec{c} \in \mathbb{F}^n$ be the point chosen by $\mathcal{V}^*$ in the sumcheck protocol above. $\mathcal{V}^*$ utilizes the verifiable polynomial delegation(VPD) to verify the value of $Q_{sim}(\vec{r}) + Z_{sim}(\vec{r})$. 

\item Finally, output the view of the simulated $\mathcal{V}^*$. 

\end{enumerate} 

Next, we need to prove this protocol is zero knowledge. Firstly, 

























