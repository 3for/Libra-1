\subsection{Discussion}\label{subsec::discuss}

In this section, we discuss some potential improvements for \name.


\paragraph{Improving verification time.} As shown in the experiments above, the verification time in \name is already fast in practice compared to other systems, yet it can be further improved by 1-2 orders of magnitude. 

Within the verification of \name, most of the time (more than 95\% in the evaluations above) is spent on our zkVPD protocols using bilinear pairings. In our current protocol, we use the pairing-based zkVPD both for the input layer and for the masking polynomials $g_i, R_i$ in each intermediate layer. Although the masking polynomials are small, the verification of our zkVPD still requires $O(s_i)$ pairings per layer for $g_i$, which is asymptotically the same as the input layer. For example, for the SHA256 circuit with 12 layers, the zkVPD verification of each $g_i$ is around 46ms, $\frac{1}{16}$ of the total verification time.\yupeng{update numbers}

However, there are many zkVPD candidates for these masking polynomials. Recall that the size of $g_i$ is only $O(s_i)$, logarithmic on the size of the circuit. We could use any zkVPD with up to linear commitment size, prover time, proof size and verification time while still maintaining the asymptotic complexity of \name. The only property we need is zero knowledge. Therefore, we can replace our pairing-based zkVPD with any of the zero knowledge proof systems we compare with as a black-box. Ligero and Aurora are of particular interest as their verification requires no cryptographic operations. If we use the black-box of these two systems for the zkVPD of $g_i, R_i$, the prover time and proof size would be affected minimally, and the verification time would be improved by almost $d$ times, as only the zkVPD of the input layer requires pairings after the change. This is a 1-2 orders-of-magnitude improvement depending on the depth of the circuit. In addition, it also removes the trusted setup in the zkVPD for the masking polynomials. We plan to integrate this approach into our system when the implementations of Ligero and Aurora become available.  


\paragraph{Removing trusted setup.} After the change above, the only place that requires trusted setup is the zkVPD for the input layer. However, replacing our pairing-based zkVPD with other systems without trusted setup may affect the succinctness of our verification time on structured circuits. For example, using Ligero, Bulletproof and Aurora as a black-box would increase the verification time to $O(n)$, and using Hyrax would increase the proof size and verification time to $O(\sqrt{n})$. Using STARK may keep the same complexity, as polynomial evaluation is a special function with short description, but the prover time and memory usage is high in STARK as shown in the experiments. Designing an efficient zkVPD protocol with logarithmic proof size and verification time without trusted setup is left as an interesting future work and we believe this paper serves as an important step towards the goal of efficient succinct zero knowledge proof without trusted setup. 